\documentclass[11pt]{article}
\usepackage[margin=1in]{geometry}
\usepackage{hyperref}
\usepackage{titlesec}
\usepackage{booktabs}
\usepackage{url}
\hypersetup{colorlinks=true,linkcolor=black,urlcolor=blue}

\title{The Human Operating System Index (HOSi):\\Quantifying Coherence for Human Regulation in the Age of AI}
\author{Christopher Salvador Ponce\\
\small Collaborators: Robert Gutzwiller, Bob Beard, Shawn Banzhaf, Wanda Wright, Pearla White}
\date{Version 1.0.0 \\ \small \url{https://humanosindex.org}}

\begin{document}
\maketitle

\begin{abstract}
This paper introduces the Human Operating System Index (HOSi)---a replicable framework for quantifying human coherence across neurochemical and energetic dimensions. Built at the intersection of neuroscience, systems thinking, and spirituality, HOSi operationalizes self-regulation as the prerequisite for ethical and creative leadership in the age of AI. Using seven measurable dimensions---Self-Awareness, Connection, Purpose, Health, Belief, Joy, and Stability---HOSi generates a Coherence Score that visualizes the state of an individual’s system. Early results suggest value for identifying energy leaks, predicting burnout risk, and supporting leadership development.
\end{abstract}

\section{Introduction}
Modern civilization advances technologically faster than humans can regulate. As AI scales, so does human dysregulation. Before optimizing machines, we must regulate humans. Dysregulated humans build dysregulated systems. HOSi provides a measurable model of human coherence via survey, reflection, and visual feedback.

\section{Theoretical Framework}
Humans seek equilibrium across neurochemical (dopamine, serotonin, oxytocin, cortisol, GABA) and energetic systems. We frame the Human Operating System as Biological Hardware, Cognitive Software, and a Spiritual Network.

\subsection{The 7 Coherence Dimensions}
\begin{center}
\begin{tabular}{@{}lllp{5.5cm}@{}}
\toprule
\# & Dimension & Primary Axis & Core Function \\
\midrule
1 & Know Yourself \& Wiring & GABA & Self-awareness, metacognition \\
2 & Family \& Friends & Oxytocin & Bonding, belonging, trust \\
3 & Mission-Driven Work & Dopamine & Motivation, flow, achievement \\
4 & Health Factors & Cortisol & Sleep, nutrition, movement \\
5 & Higher Belief \& Meaning & Serotonin & Faith, awe, optimism \\
6 & Hobbies \& Passions & Endorphins & Joy, play, expression \\
7 & Stability \& Security & Adrenaline & Safety, financial/emotional grounding \\
\bottomrule
\end{tabular}
\end{center}
Each is rated 1--10 and combined into an Overall Coherence Score.

\section{The HOSi Assessment}
A web-based tool outputs a radar-style Coherence Map and Score, with tailored reflections.

\section{Reflection \& Micro-Interventions}
Brief, testable actions support regulation (e.g., structured social contact, scheduled play, sleep hygiene).

\section{Research \& Validation Pathway}
Pilot: N$\approx$200, reliability (Cronbach's $\alpha$), correlations, pre/post after four weeks, qualitative thematics.

\section{Applications}
Individuals (personal regulation), organizations (burnout/alignment), and AI ethics (Human–AI Coherence Systems).

\section{Discussion}
HOSi pivots from performance optimization to regulation optimization and motivates integration of physiological signals (HRV, sleep, cortisol).

\section{Conclusion}
By measuring coherence across neurochemical and energetic systems, HOSi supports ethical leadership and sustainable performance: when humans achieve coherence, systems achieve harmony.

\section*{Citation}
Ponce, C. S., et al. (2025). \textit{The Human Operating System Index (HOSi): Quantifying Coherence for Human Regulation in the Age of AI}. \url{https://humanosindex.org}

\end{document}
